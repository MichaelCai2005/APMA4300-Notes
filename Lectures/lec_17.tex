\lecture{17}{1 Apr 2025}{10:10}
\section{Richardson's Extrapolation}
\begin{definition}
    Richardson extrapolation is used to construct higher order 
    approximations for lower order approximations
\end{definition}
For the setup, we approximate \(ff\) with 
\(f_1 (h)\) of the form 
\[
    f = f_1 (h) K_1 h + k_2 h^{2}  + \dots 
\]  
where the \(K\)s don't depend on \(x\). With this scheme, 
we can construct higher order approximations. For example given that 
\[
    f = f_1 (\frac{h}{2}) + K_1 (\frac{h}{2}) + K_2 (\left( 
        \frac{h}{2}
     \right)^{2}  )
\]
THen subtracting twice of this and the second we have 
\[
    f = 2 f_1 (\frac{h}{2}) - f_1 (h) - K^{2} \frac{h^{2} }{2}
+ \mathcal{O} (h^{3})
\]
\[
    2 f_1 (\frac{h}{2}) - f_1 (h) = 
    4 \frac{f(x+\frac{h}{2}) - f(x)}{h} - 
    \frac{f(x+h) - f(x)}{h}
\]
\[
    = \frac{4 f(x+ \frac{h}{2}) - 3f(x) - f (x+h)}{h}
\]
\[
    = \frac{1}{h} \left(  -f (x+h) + 4 f(x + \frac{h}{2}) - 3f(x) \right) 
\]