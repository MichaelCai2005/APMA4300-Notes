\lecture{2}{Jan 28 10:10}{}

\begin{remark}
    Gaussian Elimination:

    Suppose we had an \(n \times n \) system with n unknowns and n equations denoted as 
    \[
        Ax = y, A \in \mathbb{R^{n \times n}}
    \]

    \begin{eg}
        Given an example of the equation 
        \[
             y = a_0 + a_1 x + \dots + a_{m-1} x^{m-1} 
        \]
        How do we figure out the coefficients. Now suppose that we are given a set of points,
        it is quite easy to solve a matrix for the given system of equations 
        \[
            a_0 + a_1 x + \dots  + a_{n-1} x^{n-1}  = y_1 
        \]
        \[
            \dots 
        \]
        \[
            a_0 + a_1 x_n + \dots  = y_m
        \]
        This results in a matrix where we can solve for the values of the coefficients now. 
        \[
            \begin{pmatrix}
                1 &  x_1 &  x_1^{m-1}  \\
                \dots  &  &   \\
                1 &  x_m &   x_m^{m-1} \\
            \end{pmatrix} 
            \begin{pmatrix}
                 a_0 \\
                  \dots   \\
                  a_m \\
            \end{pmatrix}
            = 
            \begin{pmatrix}
                 y_0 \\
                  \dots \\
                  y_m \\
            \end{pmatrix}
        \]
        In practice, we could measure data points which leads to a situation where there are more equations or \(x\) values
        than coefficients which is defined as a "overdetermined matrix". On the other hand, we can also have 
        an undertermined problem. 
    \end{eg}

    \begin{eg}
        Suppose we have another example with a system of equations with \(y= a_0 
        + a_1 \sin  x + a_2 \sin  2x \dots a_n \sin  x\). 
        If we continue this this is also a linear system of equations where the matrix of the equations will be 
        \[
            \begin{pmatrix}
                1 &  \sin  x_1 &  \sin  (m-1) x_1 \\
                \dots  &  &   \\
                1 &  \sin  x_m &   \sin  (m-1) x_m\\
            \end{pmatrix} 
            \begin{pmatrix}
                 a_0 \\
                  \dots   \\
                  a_m \\
            \end{pmatrix}
            = 
            \begin{pmatrix}
                 y_0 \\
                  \dots \\
                  y_m \\
            \end{pmatrix}
        \]
        One useful example is a gaussian mixture model where you can suerimpose a number of gaussian models to get
        the value of y. Thus, solving \(Ax = y\) gives many possible applications.
    \end{eg}

\end{remark}
Elementary Row Operations:  

\begin{definition}
    Elementary row operations can be done with multiplication. Scaling is done with the following matrix 
    \[
        S_i (p) = \begin{pmatrix}
            1 &  &   \\
             & p &   \\
             &  &  1  \\
        \end{pmatrix}
    \]
    which gives us a scaling factor on a row of the matrix. We simply see that the inverse of the matrix would be simply be 
    \( \frac{1}{p}\) for \(S_i(p)^{-1} \) 
\end{definition}

\begin{definition}
    Interchange: interchanging row \( i \) and row \( j \) of \(\vec{A} \) gives us 
    \[
        E_{ij} A = \begin{pmatrix}
            1 &  &  &  &   \\
             &  0& 1 &  &   \\
             &  1& 0 &  &   \\
             &  &  &  1&   \\
             &  &  &  & 1  \\
        \end{pmatrix} A
    \]
    This showcases a exchange of rows \( i\)  and \(j \). Note that the inverse of such a matrix is just itself.
\end{definition}

\begin{definition}
    Replacement is also a method too. (finish later)
\end{definition}

\begin{remark}
    Gaussian Elimination:

    We can start with an easy example 
    \[
        A = \begin{pmatrix}
            2 &  1&  1&   0\\
             4&  3&  3&   1\\
             8&  7&  9&   5\\
             \dots &  \dots &  \dots &\dots    \\
        \end{pmatrix}
    \]
    We can then apply a matrix \(L_1\) which is our linear operator we use to transform
    our matrix using gaussian elimination.
    
    We can apply these operations which results in 
    \[
        L_{3} L_2 L_1 Ax = L_3 L_2 L_1 y
    \]
    which will give us a upper triangular matrix. We can then solve and use backwards substitution toget our answer from 
    the upper triangular matrix. 

    Let us denote this as 
    \[
         U = L_3 L_2 L_1 A
    \]
    Thus, we can write everything as 
    \[
        U x = L^{-1} y \implies LUx = y 
    \]
    where U is always an upper triangular matrix and the L is always a lower 
    triangular matrix. This only works with elimination methods and not row swaps or replacements. 
    This can be easily solvable by having 
    \[
        LU x = y \implies Lz = y
    \]
    solve for z, do a backward substitution for \( U \) and we can easily get the coefficients for 
    \( x\) 
\end{remark}



